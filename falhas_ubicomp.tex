Os componentes que integram um sistema ubíquo são peças de software e hardware cuja confiabilidade não é garantida. Falhas podem acontecer nos mais variados segmentos do sistema. Componentes de software são peças que podem não ter sido suficientemente testadas ou cuja especificação não tenha coberto todas as excessões que seus algoritmos podem gerar. Dispositivos que dependem de baterias, como laptops e telefones celulares, também não podem ser considerados totalmente confiáveis. A interoperabilidade entre dispositivos é uma questão que também gera problemas de confiança nos sistemas ubíquos. Falhas de conectividade pelas mais diversas razões, muitas das quais o sistema não pode controlar, como um usuário que leva um dispositivo para fora da área de cobertura de sinal, reduzem a confiabilidade do sistema. Os serviços básicos de contextos, arquivos, notificações, etc. podem também ser a origem de falhas em um sistema ubíquo.

Desmembraremos as falhas que podem ocorrer em um sistema ubíquo em quatro categorias: falhas nos dispositivos, nas aplicações, na rede e nos serviços. No restante da Seção cada uma dessas categorias de falhas será coberta em maior profundidade. Ao final serão apresentadas possíveis implicações das falhas debatidas ao longo da Seção.

\subsection{falhas nos dispositivos} % (fold)
\label{sub:falhas_nos_dispositivos}

% subsection falhas_nos_dispositivos (end)

\subsection{falhas nas aplicações} % (fold)
\label{sub:falhas_nas_aplicacoes}

% subsection falhas_nas_aplicações (end)

\subsection{falhas na rede} % (fold)
\label{sub:falhas_na_rede}

% subsection falhas_na_rede (end)

\subsection{falhas nos serviços} % (fold)
\label{sub:falhas_nos_servicos}

% subsection falhas_nos_serviços (end)

\subsection{implicações das falhas} % (fold)
\label{sub:implicacoes_das_falhas}


% subsection implicaçoes_das_falhas (end)
