Os componentes que integram um sistema ubíquo são peças de software e hardware cuja confiabilidade não é garantida. Componentes de software podem não ter sido suficientemente testados ou sua especificação pode não ter coberto todas as excessões que seus algoritmos poderiam gerar, como já vimos na seção anterior. Além disso, falhas podem estar inseridas nos mais variados segmentos de um sistema ubíquo. Dispositivos que dependem de baterias, como laptops e telefones celulares, não podem ser considerados totalmente confiáveis. A interoperabilidade entre dispositivos é uma questão que também gera problemas de confiança nestes sistemas. Falhas de conectividade pelas mais diversas razões, muitas das quais o sistema não pode controlar (como um usuário que leva um dispositivo para fora da área de cobertura de sinal), também reduzem a confiabilidade do sistema. Os serviços básicos de aquisição e processamento de contexto, arquivos, notificações, etc. podem também ser a origem de falhas em sistemas ubíquos.

A seguir desmembraremos as falhas que podem ocorrer em um sistema ubíquo em quatro categorias: falhas nos \textbf{dispositivos}, nas \textbf{aplicações}, na \textbf{rede} e nos \textbf{serviços}. No restante da seção cada uma dessas categorias será coberta em profundidade. Ao final serão apresentadas possíveis implicações das falhas debatidas ao longo da seção.

\subsection{falhas nos dispositivos} % (fold)
\label{sub:falhas_nos_dispositivos}

Um sistema ubíquo geralmente é composto por uma variada gama de dispositivos: computadores \emph{desktop}, \emph{laptops}, dipositivos embarcados, sensores, \emph{smartphones}, câmeras, diplays, projetores, etc. Cada dispositivo de um sistema ubíquo pode gerar suas próprias falhas. Dispositivos móveis, por exemplo, podem ficar sem bateria, sofrer problemas com o sinal e ficar com a conectividade limitada ou nula. Como vimos anteriormente, dispositivos podem ainda, ao invés de parar de funcionar, entregar resultados errados, tipo de falha à qual nos referimos como maliciosa ou \emph{bizantina}.

% subsection falhas_nos_dispositivos (end)

\subsection{falhas nas aplicações} % (fold)
\label{sub:falhas_nas_aplicacoes}

Desenvolver software confiável é, sabidamente, um processo bastante formal e custoso, mas, ainda assim, propenso a falhas. As estapas de teste, verificação e validação chegam a representar metade do custo total do software~\cite{hailpern2002software}. Na verdade, sistemas ubíquos podem ser compostos por aplicações de prateleira que não passaram por um processo de desenvolvimento rigoroso e, embora funcionem bem sozinhas, não interoperam corretamente com outros componentes do sistema. Aplicações podem falhar por diversos motivos como bugs de programação, excessões não capturadas, erros no sistema operacional e até mal uso. Sistemas ubíquos podem também ser alvo de ataques com vírus e \emph{worms}, podendo resultar em falhas \emph{bizantinas} ou interrupção do serviço de um dispositivo

% subsection falhas_nas_aplicações (end)

\subsection{falhas na rede} % (fold)
\label{sub:falhas_na_rede}

A espinha dorsal de um sistema ubíquo é a rede de comunicação. Dispositivos estão interconectados com ou sem fio, e sistemas ubíquos devem munir-se contra falhas na rede causadas por fraqueza do sinal, dispositivos que saem da área de cobertura e tornam-se inalcançáveis, congestionamento devido ao alto tráfico de dados na rede, etc., situações estas que devem ser encaradas como normais de um sistema móvel. Um problema maior ocorre quando uma falha de rede (como um dispositivo que se torna inalcançável) é percebida como uma falha de dispositivo. A identificação automática do tipo de falha é uma questão importante nos sistemas ubíquos.

% subsection falhas_na_rede (end)

\subsection{falhas nos serviços} % (fold)
\label{sub:falhas_nos_servicos}
O sistema ubíquo oferece aos seus usuários um conjunto de serviços, os quais podem ser vistos em duas categorias: serviços básicos e adicionais. Dentre os serviços básicos que um sistema ubíquo deve fornecer estão serviços de descoberta, de nomes e de eventos. Alguns sistemas devem fornecer como básicos serviços de contexto, para aquisição e interpretação de informações contextuais no sistema, e sistemas de arquivos distribuídos, para permitir o acesso ubíquo aos dados. Falhas nos serviços podem ocorrer por \emph{bugs}, erros de sistema operacional, e incluir problemas como sensoriamento/interpretação errada de contexto e perda de eventos. Falhas nos serviços podem fazer com que o sistema ubíquo inteiro pare de funcionar.

% subsection falhas_nos_serviços (end)

\subsection{implicações das falhas} % (fold)
\label{sub:implicacoes_das_falhas}

Ambientes de computação ubíqua são focados principalmente na interface de um ambiente físico com o usuário, tentando fazer como que os dispositivos sejam ``invisíveis'' e operem com a menor interveção do mesmo. Assim, a ocorrência de uma falha neste tipo de sistema pode ser desde algo irritante, que diminuirá sua aplicabilidade, até algo perigoso, que coloque a vida de algum usuário em risco.

\subsubsection*{Dores de cabeça}

Consideremos uma casa inteligente~\cite{Kidd99} onde, assim que o morador entra, sua presença é identificada e o ambiente automaticamente configurado para ele. A temperatura é ajustada, seu canal de TV preferido é acionado em algum display, a iluminação é ajustada pelo acendimento de lâmpadas e fechamento de janelas, tudo isto de forma automática. Caso esse sistema falhe, na melhor das hipóteses, causará no morador uma irritação momentânea. Mas, por exemplo, o sistema pode errar ao identificar um morador e ativar as configurações de outro usuário, eventualmente liberando acesso restrito a informações deste. Além disso, caso as falhas sejam intermitentes, o usuário precisará procurar o origem da falha e tentar corrigí-la, o que pode ser uma tarefa árdua em um ambiente com centenas de dispositivos interconectdados para oferecer serviços em cooperação. Ainda, o fato de fazer manutenção em um determinado dispositivo pode implicar em falhas de outros dispositivos que estejam funcionando corretamente, mas dependam do primeiro para continuar fornecendo algum tipo de serviço.

\subsubsection*{Risco de morte}

Sistemas ubíquos para a área médica formam é área de aplicação da computação ubíqua onde falhas podem ser catastróficas. Basicamente, espera-se que tais sistemas monitorem pacientes, identifiquem problemas e solicitem assistência médica automaticamente. Falhas em sensores de monitoramento ou no sistema de notificação, por exemplo, podem resultar em danos severos à saúde do paciente. A vida de pacientes depende fortemente destes sistemas, logo estes devem ser altamente dependáveis.

\subsubsection*{Prejuízos e quebras de segurança}

A não identificação de uma falha no sistema é, também, fonte de perigo em um sistema ubíquo. Problemas de sensoriamento podem levar ao superaquecimento ou super-resfriamento de um ambiente de temperatura controlada, acarretando na perda de alguma cultura vegetal ou mercadoria refrigerada, por exemplo. A classificação incorreta da falha é outra fonte de problemas. Por exemplo, problemas na identificação de indivíduos pode levar tanto a quebras de segurança quanto a alarmes falsos.

\subsubsection*{Problemas de contexto}

O contexto inclui qualquer informação útil sobre as caraterísticas do ambiente e dos usuários. Podemos ter diversos dispositivos, os quais  podem falhar, para realizar a aquisição de informação contextual: câmeras, sistemas de reconhecimento de voz, sensores de movimento, de temperatura, RFIDs, etc. Informações contextuais são usadas para adaptar proativamente o ambiente ubíquo às necessidades de seus usuários. Um grande desafio aqui é aumentar a robustez na inferência de informação contextual. Diversas técnicas fundem informações contextuais capturadas por diversas fontes para construir uma visão de contexto de mais alto nível, porém nem sempre conseguem criar regras claras de inferência sobre essas abstrações. Problemas na aquisição e inferência de informações contextuais podem levar a um configuração errônea do ambiente físico e levar diversos problemas já enumerados, desde má regulagem da temperatura ambiente até quebras no sistema de segurança.

% subsection implicaçoes_das_falhas (end)



