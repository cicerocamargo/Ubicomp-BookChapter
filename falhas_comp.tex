\section{Tolerância a falhas na computação}
\label{sec:falhas_comp}
\section{Tolerância a falhas na computação}
\label{sec:falhas_comp}
\section{Tolerância a falhas na computação}
\label{sec:falhas_comp}
\section{Tolerância a falhas na computação}
\label{sec:falhas_comp}
\input{falhas_comp.tex}

Falhas em programas de computador são inevitáveis, porém as consequências dessas falhas pode ser contornadas ou, ao menos minimizadas, se o design do software tiver previsto as falhas que poderiam ocorrer. Confiabilidade e disponibilidade são características altamente desejadas em sistemas de computação, pois as pessoas dependem a cada dia mais de sistemas automatizados e informatizados. É notável o crescimento em confiabilidade na área de hardware, por exemplo, onde vimos uma grande cultura de tolerância a falhas se formar dos inseguros \emph{mainframes} a vávula, cujos proprietários sofriam, rotineiramente, com falhas de diversas origens, até os robustos \emph{laptops} que temos hoje para computação pessoal.

Na área de software, por outro lado, os processos de desenvolvimento e os produtos estão cada vez mais complexos e apresentando cada vez mais bugs. 

\subsection{Falhas, erros e defeitos} % (fold)
\label{sub:falhas_erros_e_defeitos}

% subsection falhas_erros_e_defeitos (end)

\subsection{Dependabilidade} % (fold)
\label{sub:dependabilidade}

% subsection dependabilidade (end)



Falhas em programas de computador são inevitáveis, porém as consequências dessas falhas pode ser contornadas ou, ao menos minimizadas, se o design do software tiver previsto as falhas que poderiam ocorrer. Confiabilidade e disponibilidade são características altamente desejadas em sistemas de computação, pois as pessoas dependem a cada dia mais de sistemas automatizados e informatizados. É notável o crescimento em confiabilidade na área de hardware, por exemplo, onde vimos uma grande cultura de tolerância a falhas se formar dos inseguros \emph{mainframes} a vávula, cujos proprietários sofriam, rotineiramente, com falhas de diversas origens, até os robustos \emph{laptops} que temos hoje para computação pessoal.

Na área de software, por outro lado, os processos de desenvolvimento e os produtos estão cada vez mais complexos e apresentando cada vez mais bugs. 

\subsection{Falhas, erros e defeitos} % (fold)
\label{sub:falhas_erros_e_defeitos}

% subsection falhas_erros_e_defeitos (end)

\subsection{Dependabilidade} % (fold)
\label{sub:dependabilidade}

% subsection dependabilidade (end)



Falhas em programas de computador são inevitáveis, porém as consequências dessas falhas pode ser contornadas ou, ao menos minimizadas, se o design do software tiver previsto as falhas que poderiam ocorrer. Confiabilidade e disponibilidade são características altamente desejadas em sistemas de computação, pois as pessoas dependem a cada dia mais de sistemas automatizados e informatizados. É notável o crescimento em confiabilidade na área de hardware, por exemplo, onde vimos uma grande cultura de tolerância a falhas se formar dos inseguros \emph{mainframes} a vávula, cujos proprietários sofriam, rotineiramente, com falhas de diversas origens, até os robustos \emph{laptops} que temos hoje para computação pessoal.

Na área de software, por outro lado, os processos de desenvolvimento e os produtos estão cada vez mais complexos e apresentando cada vez mais bugs. 

\subsection{Falhas, erros e defeitos} % (fold)
\label{sub:falhas_erros_e_defeitos}

% subsection falhas_erros_e_defeitos (end)

\subsection{Dependabilidade} % (fold)
\label{sub:dependabilidade}

% subsection dependabilidade (end)



Falhas em programas de computador são inevitáveis, porém as consequências dessas falhas pode ser contornadas ou, ao menos minimizadas, se o design do software tiver previsto as falhas que poderiam ocorrer. Confiabilidade e disponibilidade são características altamente desejadas em sistemas de computação, pois as pessoas dependem a cada dia mais de sistemas automatizados e informatizados. É notável o crescimento em confiabilidade na área de hardware, por exemplo, onde vimos uma grande cultura de tolerância a falhas se formar dos inseguros \emph{mainframes} a vávula, cujos proprietários sofriam, rotineiramente, com falhas de diversas origens, até os robustos \emph{laptops} que temos hoje para computação pessoal.

Na área de software, por outro lado, os processos de desenvolvimento e os produtos estão cada vez mais complexos e apresentando cada vez mais bugs. 

\subsection{Falhas, erros e defeitos} % (fold)
\label{sub:falhas_erros_e_defeitos}

% subsection falhas_erros_e_defeitos (end)

\subsection{Dependabilidade} % (fold)
\label{sub:dependabilidade}

% subsection dependabilidade (end)

