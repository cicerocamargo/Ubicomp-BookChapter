A área de tolerância a falhas tem evoluído através de décadas de pesquisa nas mais diversas áreas como sistemas operacionais, redes de computadores, arquitetura de computadores, etc. Contudo, cada nova ramificação da computação propõe um novo cenário com um novo conjunto de desafios, fazendo com que as técnicas já desenvolvidas tenham aplicabilidade limitada. Esta seção se dedica a apresentar estratégias para a identificação e o tratamento das falhas geradas no âmbito de sistemas ubíquos.

\subsection{Identificação de Falhas} % (fold)
\label{sub:identificacao}

A identificação de falhas é uma tarefa árdua em sistemas ubíquos onde uma miríade de dispositivos pode estar conectada de maneiras complexas e escondidas do usuário. Por exemplo, para saber se um certo dispositivo parou, pode-se empregar alguma técnica de \emph{timeout} como \emph{heartbeat messages} (mensagens de ``batimento cardíaco''). Um dispositivo deve periodicamente enviar tais mensagens para o componente detector de falhas, de forma que este possa identificar quando um dispositivo parou de funcionar corretamente. Contudo, em um sistema com um grande número de dispositivos, as \emph{heartbeat messages} podem adicionar uma quantidade significativa de tráfego na rede e carga de trabalho no detector de falhas. Além disso, um dipositivo que se torna inalcançável na rede pode levar o detector de falhas à interpretação errônea de que o dispositivo parou de funcionar corretamente.

Falhas bizantinas, assim como em qualquer área da computação, são um problema maior ainda. Um determinado dispositivo pode, por exemplo, enviar \emph{heartbeat messages} regularmente, mas responder de forma errada as requisições. Em um sistema ubíquo, falhas bizantinas podem resultar em inferência errada de informação contextual e uso inapropriado de recursos, como exposto anteriormente.

A contenção de falhas é um assunto que ganha importância em sistemas ubíquos. Como estes sistemas podem ser compostos por uma vasta gama de dispositivos, os quais devem trocar informações com frequência, uma falha pode rapidamente propagar um erro para diversas partes do sistema. Podemos considerar espaços inteligentes que se configuram de acordo com os indivíduos que nele estão presentes: uma falha em um RFID pode levar a uma identificação errônea de um indivíduo, fazendo com que o ambiente se configure da maneira errada, por exemplo.

% subsection identificação_de_falhas (end)

\subsection{Contornando as Falhas} % (fold)
\label{sub:contornando_as_falhas}

Como dissemos na Seção~\ref{sec:falhas_comp}, os mecanismos de tolerância a falhas, em sua maioria, envolvem a replicação de algum recurso, seja hardware, sotware ou informação. Ainda temos a possibilidade de redundância temporal, onde o sistema armazena seu próprio estado de tempos em tempos para recuperar-se de alguma falha transiente. Discutiremos a seguir algumas destas técnicas de tolerância a falhas aplicadas a ambientes de computação ubíqua.

\subsubsection*{Reinício de aplicações} % (fold)

Quando uma falha é detectada em um processo, a técnica mais simples vale também em um abiente ubíquo: reiniciar o processo. Para minimizar a computação e tornar a reexecução menos perceptível ao usuário, deve-se empregar redundância temporal, ou seja, o estado do processo é armazenado periodicamente e recuperado em caso de reinício por falha. Esta técnica é bastante útil e amplamente aplicável em cenários ubíquos.

% subsubsection* reiniciar_aplicações (end)

\subsubsection*{Dispositivo suplente} % (fold)

Quando erros em uma determinada aplicação são gerados por falhas no hardware, esta aplicação deve ser escalonada em um dispositivo que forneça a mesma funcionalidade, ou funcionalidades mínimas necessárias para a aplicação. Caso o dispositivo suplente não suporte a execução da aplicação que falhou, outra aplicação que forneça funcionalidade semelhante pode ser ativada. Por exemplo, caso acabe a bateria do \emph{smartphone} do usuário e este estava tocando uma determinada música, a qual faz parte dos dados ubíquos do usuário, o sistema pode se adaptar e seguir tocando a mesma música em algum outro dispositivo presente na peça, por exemplo, o \emph{laptop} do usuário. Empregando-se redundância temporal, pode-se seguir a reprodução da música a partir de um instante próximo de onde ela havia parado no momento do término da bateria do primeiro dispositivo. Questões de disponibilidade, privacidade e segurança devem ser consideradas nestes casos.

% subsubsection* dispositivo_suplente (end)

\subsubsection*{Mecanismos alternativos de notificaçãos} % (fold)

Um ambiente ubíquo pode fonecer vários meios de entrega de informações para o usuário. Por exemplo, consideremos um ambiente médico que supervisiona pacientes e deve, automaticamente, tomar providências em caso de emergência. O sistema pode usar diversos canais de comunicação para acionar a equipe médica como uma chamada para um número de emergência ou para o telefone celular de um profissional, uma mensagem de texto via SMS, a ativação de um alarme sonoro ou visual em uma central médica, etc. Deve ser previsto um mecanismo de confimação de que a mensagem foi recebida em seu destino, para que, em caso de falha no recebimento da mensagem, o sistema possa acionar um canal de comunicação alternativo. A multiplicidade desses canais de comunicação é, por si só, um recurso simples de tolerância a falhas que aumenta significativamente a confiabilidade em um ambiente com este perfil.

% subsubsection* mecanismos_de_notificação_alternativos (end)

\subsubsection*{Correções na informação contextual} % (fold)

Tanto a aquisição quanto o processamento de informações contextuais podem ser fontes de falhas complicadas de se corrigir. Novamente, a maneira mais eficaz de resolver erros de contexto de maneira automática é empregando redundância. Podemos, por exemplo, empregar vários sensores capturando informações na mesma fonte e usar um sistema de votação para decidir qual é a informação correta, o que soluciona até mesmo falhas bizantinas nos dispositivos de sensoriamento. De forma análoga, podemos empregar diversos algoritmos para realizar múltiplas inferências sobre as mesmas informações de contexto e contornar alguma inferência errada. Contudo, dadas as técnicas desenvolvidas até o momento, a melhor maneira de detectar e corrigir erros em informação contextual ainda é permitir que o usuário corrija manualmente os erros que perceber, como no sistema apresentado em~\cite{dey2002distributed}. Esta estratégia, no entanto, requer o planejamento de uma interface com o usuário que seja a mais intuitiva e não intrusiva possível.

% subsubsection* correções_na_informação_contextual (end)

\subsubsection*{N versões} % (fold)

A redundância de software é uma estratégia igualmente aplicável em ambientes ubíquos. Genericamente, em um sistema de software com \emph{n versões} cada módulo de software possui \emph{n} algoritmos diferentes implementados para realizar a mesma tarefa. A execução das \emph{n} versões pode ser sequencial ou paralela e seu resultado é enviado para outro módulo que emprega algum sistema de votação para escolher a respota correta. Podemos estender um pouco o exemplo do parágrafo anterior, quando falamos em redundância nos módulos de inferência de contexto, para um ambiente de monitoramento médico. Podemos ter diferentes dispositivos de sensoriamento captando diferentes informações do paciente, por exemplo, câmeras, microfones, sensores fixados no corpo do indivíduo, e diferentes módulos inferindo o estado do paciente a partir destas informações. Novamente temos a necessidade de um módulo para receber estas \emph{n} informações de estado e decidir qual é o real estado do paciente, o que pode ser feito através de uma segunda inferência sobre as informações contextuais já inferidas.

% subsubsection* n_versões (end)

\subsection{Reportando falhas} % (fold)
\label{sub:reportando_falhas}

Há situações em que uma falha não pode ser contornada sem intervenção de um usuário do sistema. Nestes casos o sistema deve prover um mecanismo de relatório de falhas ao usuário, de forma que este possa tomar a decisão cabida. Este mecanismo deve ser o menos intrusivo possível, ou seja, deve desviar o usuário o mínimo possível de sua real tarefa. Falhas podem ser reportadas por diversos dispositivos como displays, alto-falantes e outro meios perceptíveis pelo usuário. Caso o indivíduo use apenas um pequeno conjunto de dispositivos, o mecanismo de relatório de falhas deve restringir-se a estes dispositivos no momento de reportar estas falhas. Contudo esse pequeno conjunto de dispositivos pode depender de diversos outros, gerando um grafo de dependências onde cada dispositivo ou serviço deste grafo pode também falhar. Falhas em certas entidades deste grafo podem não ser de responsabilidade do usuário, o que deve ser considerado quando da notificação de falhas a este. Determinar o grafo de dependências e como ele deve ser manipulaod é um importante desafio de pesquisa.

% subsection notificando_falhas (end)