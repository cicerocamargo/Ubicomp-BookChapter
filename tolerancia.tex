A área de tolerância a falhas tem evoluído através de décadas de pesquisa nas mais diversas áreas como sistemas operacionais, redes de computadores, arquitetura de computadores, etc. Contudo, cada pequena nova área da computação propõe um novo cenário com novos desafios, fazendo com que as técnicas já desenvolvidas tenham aplicabilidade limitada. Esta seção apresenta estratégias para a identificação e o tratamento das falhas geradas no contexto de sistemas ubíquos.

\subsection{Identificação de Falhas} % (fold)
\label{sub:identificacao}

A identificação de falhas é uma tarefa árdua em sistemas ubíquos onde uma miríade de dispositivos pode estar conectada de forma complexas e escondidas do usuário. Por exemplo, para saber se um certo dispositivo parou, pode-se empregar alguma técnica de \emph{timeout} como \emph{heartbeat messages} (mensagens de ``batimento cardíaco''). Um dispositivo deve periodicamente enviar tais mensagens para o componente detector de falhas, de forma que este possa identificar quando um dispositivo pare de funcionar corretamente. Contudo, em um sistema com um grande número de dispositivos, \emph{heartbeat messages} podem adicionar significativos tráfego na rede e carga de trabalho no detector de falhas. Além disso, um dipositivo que se torna inalcançável na rede pode levar o detector de falhas a uma interpretação errônea, de que o dispositivo parou de funcionar corretamente.

Falhas bizantinas, assim como em qualquer área da computação, são um problema maior ainda. Nesta classe de falhas, dispositivos que não param de funcionar (por exemplo, seguem enviando \emph{heartbeat messages} regularmente), mas começam a responder de forma errada as requisições. Em um sistema ubíquo, falhas bizantinas podem resultar em interpretação errada de informação contextual e uso inapropriado de recursos, por exemplo.



% subsection identificação_de_falhas (end)

\subsection{Mecanismos de Contenção} % (fold)
\label{sub:mecanismos_de_contencao}

% subsection mecanismos_de_contenção (end)

\subsection{Contornando as Falhas} % (fold)
\label{sub:contornando_as_falhas}

\paragraph{redundância} % (fold)
\label{par:redundancia}

Essencialmente, todos os mecanismos de tolerância a falhas envolvem a replicação de algum recurso.

% paragraph redundância (end)

% subsection contornando_as_falhas (end)

\subsection{Notificando falhas} % (fold)
\label{sub:notificando_falhas}

Há situações em que uma falha não pode ser contornada sem intervenção de um usuário do sistema. Nestes casos o sistema deve prover um mecanismo de relatório de falhas ao usuário, de forma que este possa tomar a decisão cabida. Este mecanismo deve ser o menos intrusivo possível (...)

% subsection notificando_falhas (end)
