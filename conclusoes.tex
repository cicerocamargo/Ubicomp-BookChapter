Tolerância a falhas é um assunto ainda pouco maduro na área da computação ubíqua, em parte devido ao fato da computação ubíqua em si ser relativamente nova. Contudo, avanços significativos têm sido alcançados, alguns desses avanços aparecem como pequenos mecanismos em aplicações específicas de computação ubíqua que necessitam um ou outro recurso de tolerância a falhas. A maioria dos trabalhos revisados busca identificar um conjunto de técnicas básicas de tolerância a falhas que devem estar presentes na maioria dos sistemas de computação ubíqua, e empacotálas em forma de um \emph{framework} ou um \emph{middleware} para suporte ao desenvolvimento de aplicações ubíquas.

A área da computação ubíqua tem um perfil mais focado na interface do sistema com o usuário final e na incremental fusão de elementos de computação com ambiente físico. Contudo, para que a ubiquidade seja viável é primordial que os sistemas sejam tolerantes a falhas, pois as pessoas dependerão destes sistemas para tarefas complexas e, muitas vezes, vitais. As ferramentas e técnicas existentes de tolerância a falhas necessitam ainda de um considerável amadurecimento, que certamente se acentuará conforme houver mais demanda comercial por soluções de computação ubíqua.