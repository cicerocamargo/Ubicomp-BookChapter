Diversos trabalhos têm surgido na academia propondo aplicações reais de computação ubíqua. Cada uma possui suas peculiaridades e apresentam um novo ambiente de aplicação com seu modelo de falhas característico. \emph{A ferramenta x, por exemplo, ... Já a ferramenta y ...}

Este capítulo tem como objetivo apresentar estudos de caso da academia onde soluções gerais e específicas são empregadas no sentido de tolerar falhas nestes sistemas. Na Subseção~\ref{sub:cobra} é apresentada a arquitetura de software CoBrA, idealizada como infraestrutua para ambientes inteligentes. (...)

\subsection{CoBrA} % (fold)
\label{sub:cobra}

% subsection cobra (end)

\subsection{Coda FS} % (fold)
\label{sub:coda_fs}

% subsection coda_fs (end)

\subsection{GaiaOS} % (fold)
\label{sub:gaiaos}

% subsection gaiaos (end)

\subsection{Prism-MW} % (fold)
\label{sub:prism_mw}

O Prism-MW~\cite{Seo07} é uma plataforma de \emph{middleware} que fornece uma arquitetura baseada em componentes básicos de construção de sistemas ubíquos que podem ser mapeados diretamente em implementações reais. O \emph{middleware} arquitetural é, basicamente composto por 3 camadas. Na base uma Máquina Virtual, que permite a implantação do sistema em plataformas hetereogêneas sem modificações na implementação. Acima da máquina virtual são fornecidos os blocos básicos de construção da arquitetura (componentes, conectores, eventos, etc.). No topo, usando os componentes da camada arquitetural, são implementados três serviços considerados  para sistemas ubíquos tolerantes a falhas: descoberta dinâmica de novos serviços e recursos, recuperação automática e transparente de falhas e determinação analítica das estratégias de replicação de componentes e arquiteturas de implantação.

% subsection prism_mw (end)