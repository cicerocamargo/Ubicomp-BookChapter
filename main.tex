\documentclass{SBCbookchapter}
\usepackage[utf8x]{inputenc}
\usepackage[T1]{fontenc}
\usepackage[brazilian]{babel}
\usepackage{graphicx}
\usepackage{hyperref}

\hyphenpenalty=5000
\tolerance=1000


\title{Tolerância a Falhas na Computação Ubíqua}
\author{Cícero Camargo}

\begin{document}
\maketitle

% \begin{abstract}
% The book is on the table
% \end{abstract}

\begin{resumo}

O conceito de computação ubíqua abre um leque de possíveis aplicações, nas quais diversos recursos computacionais estarão distribuídos e integrados no ambiente físico. Estas características da ubiquidade complicam ainda mais as questões relativas a tolerância a falhas. No entanto, diversas aplicações tem sido propostas nos moldes da computação ubíqua em cenários de aplicação onde falhas podem representar desde um entrave na usabilidade do sistema até um perigo para a vida humana, como em aplicações ubíquas para a área médica. A tolerância a falhas no contexto da computação ubíqua é uma área ainda recente e este trabalho tem como proposta revisar técnicas já consolidadas na academia, bem como o estado da arte na área.

\end{resumo}

\tableofcontents

\section{Introdução} % (fold)

A computação ubíqua anuncia uma nova era, a qual já começamos a vivenciar, onde diversos dispositivos computacionais são completamente integrados com o ambiente e interconectados de forma que os usuários podem acessar seus dados e aplicações com diversos níveis de transparência. A sustentabilidade da computação pervasiva depende que a complexidade da infraestrutura de software e hardware esteja escondida do usuário final. Em um ambiente onde dispositivos de rede, computadores pessoais, sensores, entre outros dispositivos, devem ser integrados de forma transparente ao usuário, as falhas que podem ocorrer são exemplos de complexidades que devem ser abstraídas ou contornadas sempre que possível.

Tolerância a falhas é um tema ainda pouco abordado no âmbito dos sistemas ubíquos, porém é um tema primordial para que acomputação ubíqua ganhe espaço, pois uma vez que tais sistemas devem residir no mesmo meio físico (e virtual) dos usuários, falhas podem ser irritantes e, até mesmo, uma ameaça a vidas humanas. Diversos pesquisadores da área argumentam que as técnicas atuais de tolerância a falhas devem ser repensadas para o contexto ubíquo. Em~\cite{Banavar2000} o autor reafirma a necessidade de proteger um sistema contra falhas para ser aplicado em casas inteligentes. Discussão semelhante é apresentada em~\cite{Bohn02} onde o autor expressa os requisitos de tolerância a falhas específicos de ambientes de monitoramento na área médica.

A computação ubíqua tem encontrado uma aplicação direta na área médica\linebreak\cite{Bang03} e em espaços inteligentes (\emph{smart spaces})~\cite{Kidd99} onde sensores são usados para monitorar pacientes em um hospital, identificar distúrbios da idade em idosos ou capturar o estado de instalações elétricas em cômodos inteligentes. Neste cenário de aplicação, a ocorrência de falhas pode accaretar em sérias consequências, até mesmo em perda de vidas. Logo, tolerância a falhas é uma questão vital.

As falhas também abrem novas situações com diferentes implicações no contexto de sistemas ubíquos. Por exemplo, falhas podem resultar em informações de contexto incorretas, levando a falhas de segurança, privacidade e acesso indevido de recursos. Logo, a contenção de falhas é um aspecto de grande importância em um sistema ubíquo para uso comercial.

O objetivo deste capítulo é apresentar os principais desafios relacionados a tolerância a falhas dentro da computação ubíqua, bem como apresentar diversas estratégias já propostas e, algumas, implementadas em ambientes e ferramentas de computação ubíqua desenvolvidas por grupos de pesquisa.

Em ~\emph{``\nameref{sec:falhas_comp}''} serão introduzidos diversos conceitos consolidados sobre tolerância a falhas na computação como um todo. \emph{``\nameref{sec:falhas_ubicomp}''} apresenta os principais desafios de tolerância a falhas encontrados pelos pesquisadores da área de computação ubíqua em específico. Em \emph{``\nameref{sec:tolerancia}''} são apresentados alguns avanços de pesquisa em direção a uma computação ubíqua tolerante a falhas. A Seção~\emph{``\nameref{sec:estudos}''} mostra o que tem sido aplicado de soluções práticas de tolerância a falhas em ferramentas e ambientes de computação ubíqua. Em \emph{``\nameref{sec:conclusoes}''} é discutido o estado atual desta subárea e os avanços críticos em tolerância a falhas que, acredita-se, devam surgir para que a computação ubíqua avance como um todo e mais soluções apareçam no mercado, para o usuário final.

% section introducao (end)

\section{Tolerância a falhas na computação}
\label{sec:falhas_comp}
\section{Tolerância a falhas na computação}
\label{sec:falhas_comp}
\section{Tolerância a falhas na computação}
\label{sec:falhas_comp}
\section{Tolerância a falhas na computação}
\label{sec:falhas_comp}
\input{falhas_comp.tex}

Falhas em programas de computador são inevitáveis, porém as consequências dessas falhas pode ser contornadas ou, ao menos minimizadas, se o design do software tiver previsto as falhas que poderiam ocorrer. Confiabilidade e disponibilidade são características altamente desejadas em sistemas de computação, pois as pessoas dependem a cada dia mais de sistemas automatizados e informatizados. É notável o crescimento em confiabilidade na área de hardware, por exemplo, onde vimos uma grande cultura de tolerância a falhas se formar dos inseguros \emph{mainframes} a vávula, cujos proprietários sofriam, rotineiramente, com falhas de diversas origens, até os robustos \emph{laptops} que temos hoje para computação pessoal.

Na área de software, por outro lado, os processos de desenvolvimento e os produtos estão cada vez mais complexos e apresentando cada vez mais bugs. 

\subsection{Falhas, erros e defeitos} % (fold)
\label{sub:falhas_erros_e_defeitos}

% subsection falhas_erros_e_defeitos (end)

\subsection{Dependabilidade} % (fold)
\label{sub:dependabilidade}

% subsection dependabilidade (end)



Falhas em programas de computador são inevitáveis, porém as consequências dessas falhas pode ser contornadas ou, ao menos minimizadas, se o design do software tiver previsto as falhas que poderiam ocorrer. Confiabilidade e disponibilidade são características altamente desejadas em sistemas de computação, pois as pessoas dependem a cada dia mais de sistemas automatizados e informatizados. É notável o crescimento em confiabilidade na área de hardware, por exemplo, onde vimos uma grande cultura de tolerância a falhas se formar dos inseguros \emph{mainframes} a vávula, cujos proprietários sofriam, rotineiramente, com falhas de diversas origens, até os robustos \emph{laptops} que temos hoje para computação pessoal.

Na área de software, por outro lado, os processos de desenvolvimento e os produtos estão cada vez mais complexos e apresentando cada vez mais bugs. 

\subsection{Falhas, erros e defeitos} % (fold)
\label{sub:falhas_erros_e_defeitos}

% subsection falhas_erros_e_defeitos (end)

\subsection{Dependabilidade} % (fold)
\label{sub:dependabilidade}

% subsection dependabilidade (end)



Falhas em programas de computador são inevitáveis, porém as consequências dessas falhas pode ser contornadas ou, ao menos minimizadas, se o design do software tiver previsto as falhas que poderiam ocorrer. Confiabilidade e disponibilidade são características altamente desejadas em sistemas de computação, pois as pessoas dependem a cada dia mais de sistemas automatizados e informatizados. É notável o crescimento em confiabilidade na área de hardware, por exemplo, onde vimos uma grande cultura de tolerância a falhas se formar dos inseguros \emph{mainframes} a vávula, cujos proprietários sofriam, rotineiramente, com falhas de diversas origens, até os robustos \emph{laptops} que temos hoje para computação pessoal.

Na área de software, por outro lado, os processos de desenvolvimento e os produtos estão cada vez mais complexos e apresentando cada vez mais bugs. 

\subsection{Falhas, erros e defeitos} % (fold)
\label{sub:falhas_erros_e_defeitos}

% subsection falhas_erros_e_defeitos (end)

\subsection{Dependabilidade} % (fold)
\label{sub:dependabilidade}

% subsection dependabilidade (end)



\section{Falhas em Ambientes Ubíquos}
\label{sec:falhas_ubicomp}
Os componentes que integram um sistema ubíquo são peças de software e hardware cuja confiabilidade não é garantida. Componentes de software são peças que podem não ter sido suficientemente testadas ou cuja especificação não tenha coberto todas as excessões que seus algoritmos podem gerar, como já vimos na seção anterior. Além disso, falhas podem estar inseridas nos mais variados segmentos de um sistema ubíquo. Dispositivos que dependem de baterias, como laptops e telefones celulares, não podem ser considerados totalmente confiáveis. A interoperabilidade entre dispositivos é uma questão que também gera problemas de confiança nestes sistemas. Falhas de conectividade pelas mais diversas razões, muitas das quais o sistema não pode controlar (como um usuário que leva um dispositivo para fora da área de cobertura de sinal), reduzem a confiabilidade do sistema. Os serviços básicos de contextos, arquivos, notificações, etc. podem também ser a origem de falhas em sistemas ubíquos.

A seguir desmembraremos as falhas que podem ocorrer em um sistema ubíquo em quatro categorias: falhas nos dispositivos, nas aplicações, na rede e nos serviços. No restante da Seção cada uma dessas categorias de falhas será coberta em profundidade. Ao final serão apresentadas possíveis implicações das falhas debatidas ao longo da seção.

\subsection{falhas nos dispositivos} % (fold)
\label{sub:falhas_nos_dispositivos}

Um sistema ubíquo geralmente é composto por uma variada gama de dispositivos: computadores \emph{desktop}, \emph{laptops}, dipositivos embarcados, sensores, \emph{smartphones}, câmeras, diplays, projetores, etc. Cada dispositivo de um sistema ubíquo pode gerar suas próprias falhas. Dispositivos móveis, por exemplo, podem ficar sem bateria, sofrer problemas com o sinal e ficar com a conectividade limitada ou nula. Como vimos anteriormente, dispositivos podem ainda, ao invés de parar de funcionar, entregar resultados errados, tipo de falha à qual nos referimos como \emph{falha bizantina}.

% subsection falhas_nos_dispositivos (end)

\subsection{falhas nas aplicações} % (fold)
\label{sub:falhas_nas_aplicacoes}

Desenvolver software confiável é, sabidamente, um processo mais formal e custoso, e, ainda assim, propendo a falhas. As estapas de teste, verificação e validação chegam a representar metade do custo total do software~\cite{hailpern2002software}. Na verdade, sistemas ubíquos são compostos por aplicações de prateleira que não passaram por um processo de desenvolvimento rigoroso e, embora funcionem bem sozinhas não interoperam bem com outros componentes de software. Aplicações podem falhar por diversos motivos como bugs de programação, excessões não capturadas, erros no sistema operacional e até mal uso. Sistemas ubíquos podem também ser alvo de ataques com vírus e \emph{worms}, podendo resultar em falhas \emph{bizantinas} ou interrompe o serviço de um dispositivo

% subsection falhas_nas_aplicações (end)

\subsection{falhas na rede} % (fold)
\label{sub:falhas_na_rede}

A espinha dorsal de um sistema ubíquo é a rede de comunicação. Dispositivos estão interconectados com ou sem fio, e sistemas ubíquos devem se preparar contra falhas na rede causadas por fraqueza do sinal, ou dispositivos que saem da área de cobertura e tornam-se inalcançáveis, ou congestionamento devido ao alto tráfico de dados na rede, etc., situações que devem ser encaradas como normais de um sistema móvel. Um problema maior ocorre quando uma falha de rede (como um dispositivo que se torna inalcançável) é percebida como uma falha de dispositivo. A questão de identificação automática do tipo de falha é uma questão importante nos sistemas ubíquos.

% subsection falhas_na_rede (end)

\subsection{falhas nos serviços} % (fold)
\label{sub:falhas_nos_servicos}
O sistema ubíquo oferece aos seus usuários um conjunto de serviços, os quais podem ser vistos em duas categorias: serviços básicos e adicionais. Dentre os serviços básicos que um sistema ubíquo deve fornecer estão serviços de descoberta, de nomes e de eventos. Alguns sistemas devem fornecer como básicos serviços de contexto, para aquisição e interpretação de informações contextuais no sistema, e sistemas de arquivos distribuídos, para permitir o acesso ubíquo aos dados. Falhas nos serviços podem ocorrer por \emph{bugs}, erros de sistema operacional, e incluir problemas como sensoriamento e interpretação errada de contexto e perda de eventos. Falhas nos serviços podem fazer com que o sistema ubíquo inteiro pare de funcionar.

% subsection falhas_nos_serviços (end)

\subsection{implicações das falhas} % (fold)
\label{sub:implicacoes_das_falhas}

Ambientes de computação ubíqua são focados principalmente na interface de um ambiente, tentam fazer como que os dispositivos sejam ``invisíveis'' no ambiente e que estes operem com a menor interveção do usuário. Assim, a ocorrência de falhas neste perfil de sistem pode ser algo irritante e diminuir muito sua aplicabilidade.

\subsubsection*{Dores de cabeça}

Por exemplo, em uma casa inteligente~\cite{Kidd99} onde, assim que o morador entra, sua presença é logo identificada e o ambiente é ajustado para ele. A temperatura é ajustada automaticamente, seu canal de TV preferido é acionado em algum display, a iluminação é ajustada pelo acendimento automático de lâmpadas e fechamento automático de janelas. Caso esse sistema falhe, na melhor das hipóteses, causará no morador uma irritação momentânea, por exemplo, o sistema pode errar ao identificar um morador e ativar as configurações de outro. Porém, caso as falhas sejam intermitentes, estas causarão um aborrecimento maior, pois o morador precisará procurar o origem da falha e tentar corrigi-la, o que pode ser uma tarefa árdua em um ambiente com centenas de dispositivos interconectdados para oferecer serviços em cooperação. Além disso, o fato de fazer manutenção em um determinado dispositivo pode implicar na falha de outros dispositivos que dependam do primeiro para continuar servindo um serviço qualquer.

\subsubsection*{Riscos de vida}

Sistemas ubíquos para a área médica formam uma área de aplicação onde falhas podem ser catastróficas. Basicamente, espera-se que tais sistemas monitorem pacientes, identifiquem problemas e solicitem assistência médica automaticamente. Falhas em sensores de monitoramento ou no sistema de notificação, por exemplo, podem resultar em danos severos à saúde do paciente. A vida de pacientes depende fortemente destes sistemas, logo estes devem ser altamente confiáveis.

\subsubsection*{Prejuízos e quebras de segurança}

A não identificação de uma falha no sistema pode ser, também, fonte de perigo em um sistema ubíquo. Problemas de sensoriamento poder levar ao superaquecimento ou super-resfriamento de um ambiente de temperatura controlada, acarretando na perda de alguma cultura vegetal ou mercadoria refrigerada, por exemplo. A classificação incorreta da falha é outra fonte de problemas. Por exemplo, problemas na identificação de indivíduos pode levar tanto a quebras de segurança quanto a alarmes falsos.

\subsubsection*{Problemas de contexto}

O contexto inclui qualquer informação útil sobre as caraterísticas do ambiente e dos usuários, assim podemos ter diversos dispositivos (que podem falhar) para aquisição de informação contextual: câmeras, sistemas de reconhecimento de voz, sensores de movimento, de temperatura, RFIDs, etc. Informações contextuais são usadas para adaptar proativamente o ambiente ubíquo às necessidades de seus usuários. Um grande desafio de tolerância a falhas é inferir o contexto corretamente. Diversas técnicas fundem informações contextuais capturadas para contruir uma visão de contexto mais abstrata, porém nem sempre conseguem criar regras claras de inferência sobre essas abstrações. Problemas na aquisição e inferência de informações contextuais podem levar a um configuração erronea do ambiente físico e levar diveros problemas já enumerados, desde uma temperatura ambiente mal regulada até a quebras no sistema de segurança.

% subsection implicaçoes_das_falhas (end)





\section{Abordagens para Tolerância a falhas em ambientes ubíquos}
\label{sec:tolerancia}
A área de tolerância a falhas tem evoluído através de décadas de pesquisa nas mais diversas áreas como sistemas operacionais, redes de computadores, arquitetura de computadores, etc. Contudo, cada pequena nova área da computação propõe um novo cenário com novos desafios, fazendo com que as técnicas já desenvolvidas tenham aplicabilidade limitada. Esta seção apresenta estratégias para a identificação e o tratamento das falhas geradas no contexto de sistemas ubíquos.

\subsection{Identificação de Falhas} % (fold)
\label{sub:identificacao}

A identificação de falhas é uma tarefa árdua em sistemas ubíquos onde uma miríade de dispositivos pode estar conectada de forma complexas e escondidas do usuário. Por exemplo, para saber se um certo dispositivo parou, pode-se empregar alguma técnica de \emph{timeout} como \emph{heartbeat messages} (mensagens de ``batimento cardíaco''). Um dispositivo deve periodicamente enviar tais mensagens para o componente detector de falhas, de forma que este possa identificar quando um dispositivo pare de funcionar corretamente. Contudo, em um sistema com um grande número de dispositivos, \emph{heartbeat messages} podem adicionar significativos tráfego na rede e carga de trabalho no detector de falhas. Além disso, um dipositivo que se torna inalcançável na rede pode levar o detector de falhas a uma interpretação errônea, de que o dispositivo parou de funcionar corretamente.

Falhas bizantinas, assim como em qualquer área da computação, são um problema maior ainda. Nesta classe de falhas, dispositivos que não param de funcionar (por exemplo, seguem enviando \emph{heartbeat messages} regularmente), mas começam a responder de forma errada as requisições. Em um sistema ubíquo, falhas bizantinas podem resultar em interpretação errada de informação contextual e uso inapropriado de recursos, por exemplo.



% subsection identificação_de_falhas (end)

\subsection{Mecanismos de Contenção} % (fold)
\label{sub:mecanismos_de_contencao}

% subsection mecanismos_de_contenção (end)

\subsection{Contornando as Falhas} % (fold)
\label{sub:contornando_as_falhas}

\paragraph{redundância} % (fold)
\label{par:redundancia}

Essencialmente, todos os mecanismos de tolerância a falhas envolvem a replicação de algum recurso.

% paragraph redundância (end)

% subsection contornando_as_falhas (end)

\subsection{Notificando falhas} % (fold)
\label{sub:notificando_falhas}

Há situações em que uma falha não pode ser contornada sem intervenção de um usuário do sistema. Nestes casos o sistema deve prover um mecanismo de relatório de falhas ao usuário, de forma que este possa tomar a decisão cabida. Este mecanismo deve ser o menos intrusivo possível (...)

% subsection notificando_falhas (end)


\section{Estudos de Caso}
\label{sec:estudos}
Diversos trabalhos têm surgido na academia propondo aplicações práticas de computação ubíqua. No entanto, cada domínio de aplicação possui suas peculiaridades e apresenta seu modelo de falhas característico. Esta seção destina-se a apresentar estudos de caso da academia onde soluções gerais e específicas são empregadas no sentido de tolerar falhas nestes sistemas.

\subsection{CoBrA} % (fold)
\label{sub:cobra}

CoBrA (\emph{\textbf{C}ontext \textbf{Br}oker \textbf{A}rchiecture}) é um \emph{middleware} de computação ubíqua que fornece serviços básicos para a aquisição e gerenciamento de informação contextual em espaços inteligentes (\emph{smart spaces}). O objetivo de CoBrA é permitir que agentes distribuídos em um ambiente ubíquo possam acessar um modelo compartilhado de contextos, contribuir com informações para este modelo e gerenciar o acesso de terceiros a suas informações contextuais em um ambiente ubíquo com sensibilidade o contexto.

CoBrA possui uma arquitetura \emph{broker-centric}, ou seja, existe um agente central (\emph{broker}) que faz inferências sobre as informações contextuais dos demais agentes do ambiente e serve como intermediário de todas as trocas de informações contextuais entre os demais agentes. Podemos ainda ter redes de \emph{brokers} trocando informações contextuais sobre os agentes presentes em seus respectivos espaços inteligentes, como é o caso do sistema ubíquo EasyMeeting~\cite{finin2005semantic}, ilustrado na Figura~\ref{fig:_cobra_easy_meeting}, sistema este construído sobre o CoBrA.

\begin{figure}[htbp]
	\centering
		\includegraphics[width=.7\textwidth]{figuras/cobra_easy_meeting.pdf}
	\caption{EasyMeeting: estudo de caso desenvolvido sobre CoBrA.}
	\label{fig:_cobra_easy_meeting}
\end{figure}

CoBrA propõe mecanismos de tolerância a falhas baseados na arquitetura proposta em~\cite{kumar2000adaptive} sobre \emph{times persistentes de brokers}. Tal arquitetura proposta (\emph{Adaptive Agent Architecture}) assume que para um \emph{broker} ingressar em um time, este deve assumir comprometimentos em relação aos serviços que devem prestar, o que facilita a substituição de um \emph{broker} que apresenta falhas e faz com que o sistema forneça funcionalidades básicas enquanto houver pelo menos um \emph{broker} funcionando. Outros comprometimentos que os \emph{brokers} podem assumir lhes dão a permissão para criar/escolher novos \emph{brokers} e incluí-los no time, fazendo com que o sistema consiga se adaptar automaticamente perante variações no ambiente, tais como falhas nos agentes. 

A teoria que embasa estes mecanismos é a \textbf{teoria das intenções conjuntas} (\emph{theory of joint intentions}), amplamente utilizada em ramos da computação ligados à Inteligência Artificial. Este conjunto de técnicas espera, intuitivamente, que um time será mais robusto do que um conjunto de indivíduos. A premissa básica é de que um time trabalha junto para atingir uma determinada meta (por exemplo, prover serviços de contexto). Quando um membro do time está com problemas outros membros do time o ajudarão (por exemplo, um servidor pode ordenar o reinício de um outro servidor que apresenta uma falha transiente e restaurar seu estado) e quando um membro ficar indisponível o restante dos membros farão seu trabalho para cumprir as metas do time (por exemplo, chamadas ao servidor indiponível podem ser atendidas por um outro servidor qualquer do time). Um time deixará de cumprir uma meta global se, e somente se, todos os membros escolherem esta opção. Logo, times são organizações inerentemente tolerantes a falhas.

Em~\cite{kumar2000adaptive} são detalhadas estratégias necessárias para implementar os mecaninsmos de times em software, inclusive os protocolos de comunicação que devem ser usados para a autocoordenação dos times de \emph{brokers}.

% subsection cobra (end)

\subsection{Coda FS} % (fold)
\label{sub:coda_fs}

O Coda File System~\cite{satyanarayanan1990coda} é um sistema de arquivos distribuído descendente do Andrew File System (AFS)~\cite{howard1988overview} e fornece diversos recursos semelhantes aos do AFS. Coda foi projetado para ser um sistema de arquivos escalável, seguro e altamente disponível, com alto grau de transparência de distribuição. Tomando também em consideração a alta disponibilidade, os projetistas do Coda tentaram chegar a um alto grau de transparência na recuperação de falhas. Para obter algum grau de transparência de recuperação de falhas os desenvlovedores do Coda criaram vários mecanismos sofisticados baseados em cache de dados no lado do cliente e replicação de servidores de arquivos.

\subsubsection*{Operação desconectada}

Os clientes Coda, ao contrário de clientes NFS, por exemplo, continuam a operar normalmente mesmo sem conseguir conectar nenhum dos servidores~\cite{kistler1992disconnected}. Uma vez que o cliente abre um arquivo do servidor, este  passa a editar uma cópia local, de forma que, quando o usuário terminar sua sessão sem conexão, não resultará em erro algum. Quando as alterações forem atualizadas no(s) servidor(es), no entanto, pode ser que ocorra algum conflito, o qual nem sempre é resolvido automaticamente. Porém raramente os usuários compartilham um arquivo para escrita, tornando esta situação uma exceção.

Para que a operação desconectada realmente funcione, todos os arquivos que o cliente precisa devem estar em cache. Coda provê um mecanismo sofisticado chamado \emph{hoarding} para garantir isso. O mecanismo funciona como segue. Primeiro, um usuário pode criar uma lista dos arquivos que o mesmo julga serem os mais importantes para ele (\emph{hoard database}). O Coda usa esta lista fornecida pelo usuário juntamente com a localidade de referência aos arquivos para atribuir prioridades a estes. Uma vez que foram atribuídas prioridades a todos os arquivos do usuário, o Coda se vale de três regras básicas para gerenciar a cache:

\begin{enumerate}
	\item Não existe nenhum arquivo fora da cache com prioridade maior que a de um que está na cache;
	\item Ou a cache está cheia ou nenhum arquivo fora da cache tem prioridade não-zero;
	\item Cada arquivo em cache é somente uma cópia de um arquivo no servidor.
\end{enumerate}

Para o Coda, se a cache satisfaz estas 3 regras ela está em estado de \emph{equilibrium}. Contudo as prioridades podem ser alteradas dinamicamente. Para reestabelecer o \emph{equilibrium} da cache, a cada 10 minutos o módulo cliente realiza o que os autores chamam de \emph{hoard walk}, reajustando prioridades e substituindo arquivos na cache. Contudo, embora estas técnicas tenham aumentado muito a eficiência do Coda FS, elas não garantem que todos os dados que o cliente precisará futuramente estarão armazenados em cache no caso de uma operação desconectada.

\subsubsection*{Memória Virtual Recuperável}
Além de prover alta disponibilidade com as operações desconectadas, Coda emprega um mecanismo que facilita a recuperação de um processo em falha. Memória Virtual Recuperável (\emph{Recoverable Virtual Memory} -- RVM) é um mecanismo que armazena estruturas de dados cruciais da a aplicação para uma recuperação rápida em caso de falha (mais detalhes em~\cite{satyanarayanan1994lightweight}).

A ideia do RVM é bastante simples. As estruturas de dados mais importantes da aplicação são mantidas em um espaço conhecido da memória principal e são armazenados logs para cada alteração nestes dados, de maneira semelhante a um esquema de transações, porém sem suporte a concorrência. Em caso de falha de um processo, os dados são restaurados de maneira relativamente fácil através do log de operações nas estruturas de dados mais importantes.

O Coda FS ainda implementa diversos mecanismos de segurança que aumentam ainda mais a confiabilidade dos usuários neste sistema de arquivo.

% subsection coda_fs (end)

\subsection{Prism-MW} % (fold)
\label{sub:prism_mw}

Prism-MW~\cite{Seo07} é uma plataforma de \emph{middleware} que oferece um conjunto de blocos básicos para a construção de sistemas ubíquos. O Prism-MW fornece mecanismos de tolerância a falhas embutidos, ao mesmo tempo que sua arquitetura é eficiente e clara. O \emph{middleware} arquitetural é, basicamente composto por 3 camadas. Na base uma Máquina Virtual, que permite a implantação do sistema em plataformas hetereogêneas sem modificações na implementação. Acima da máquina virtual são fornecidos os blocos básicos de construção da arquitetura (componentes, conectores, eventos, etc.). No topo, usando os componentes da camada arquitetural, são implementados três serviços considerados essenciais para sistemas ubíquos: $(1)$ descoberta dinâmica de novos serviços e recursos, $(2)$ recuperação automática e transparente de falhas e $(3)$ determinação analítica das estratégias de replicação de componentes e arquiteturas de implantação.

\begin{figure}[htbp]
	\centering
		\includegraphics[width=.7\textwidth]{figuras/prism.pdf}
	\caption{Visão geral do \emph{middleware} Prism.}
	\label{fig:prism}
\end{figure}

Como pode ser visto na Figura~\ref{fig:prism} a abordagem do Prism-MW consiste em separar totalmente a lógica da aplicação da lógica de tolerância a falhas, fornecendo esta última como um serviço da camada superior da arquitetura. Esta abordagem consiste basicamente em fornecer \textbf{conectores lógicos} entre os componentes, que oferecem diferentes garantias para os diferentes elementos que irão conectar-se através deles, e módulos que auxiliam no \textbf{gerenciamento de componentes replicados}. A abordagem de Prism auxilia na tarefa de construção, análise e adaptação de software ubíquo mantendo a qualidade de serviço (QoS) em diversos cenários ubíquos~\cite{Seo07}.

% subsection prism_mw (end)

% \subsection{ADRF} % (fold)
% 
% \emph{Assured Dynamic Reconfiguration Framework} (ADRF) é um \emph{framework} qua fornece recursos básicos para a completa e correta reconfiguração de um sistema ubíquo. 

% subsection adrf (end)

% \subsection{GaiaOS} % (fold)
% \label{sub:gaiaos}
% 
% O projeto Gaia consiste de uma plataforma semelhante a um sistema operacional para ambientes ubíquos, que fornece um mecanismo de abstração do acesso aos recursos físicos presentes em um espaço inteligente. Diversos conceitos presentes em sistemas operacionais modernos, como eventos, sinais, sistemas de arquivos e processos, são traduzidos no GaiaOS para o contexto da computação ubíqua. A Figura~\ref{fig:figuras_gaiaos} fornece uma visão geral dos componentes em um sistema GaiaOS. Como pode-se observar nesta ilustração, o \emph{Unified Object Bus} é a espinha dorsal do GaiaOS, pois é o barramento pelo qual todos os dispositivos em um espaço inteligente podem se intercomuncar por uma única linguagem.
% 
% \begin{figure}[htbp]
% 	\centering
% 		\includegraphics[width=.7\textwidth]{figuras/gaiaos.pdf}
% 	\caption{Visão geral do GaiaOS.}
% 	\label{fig:figuras_gaiaos}
% \end{figure}
% 
% No \emph{kernel} do GaiaOS são implementados serviços de inicialização básica do sistema: gerenciador de eventos; localizador de serviçoos e indivíduos; base de registros de dispositivos; serviços, indivíduos e aplicações ativos no sistema; sistema de arquivos; e serviços de autentitcaçãoo e autorização de usuários.
% 
% A extensão \emph{Gaia Context Infrastructure}~\cite{ranganathan2003middleware} adiciona sensibilidade ao contexto a espaços inteligentes construídos com GaiaOS.
% 
% % subsection gaiaos (end)


\subsection{Soluções de Medicina Ubíqua} % (fold)

A área médica, embora apresente uma enorme demanda por soluções de tolerância a falhas, fornece mais sugestões de soluções para os problemas possíveis em ambientes de medicina ubíqua do que protótipos de fato. O artigo ``\emph{Dependability Issues of Pervasive Computing in a Healthcare Environment}''~\cite{bohn2004dependability} levanta diversos problemas e soluções em medicina ubíqua, principalmente ligados a falhas de segurança, os quais serão discutidos a seguir.

Hospitais geralmente possuem os registros de seus pacientes em um banco de dados, contudo o mecanismo de controle de acesso ao banco de dados não é flexível o suficiente para um ambiente ubíquo, onde o acesso deve ser feito de ``qualquer lugar''. Para obter esta flexibilidade o mecanismo de \emph{controle de acesso} a esse banco de dados deve ser replicado e separado do mecanismo de \emph{gerenciamento} das informações do banco. Contudo, estes servidores de controle de acesso podem falhar. Assim, \emph{pontos de acesso} espalhados pelo hospital devem ``alcançar'' um número $k$ (maior ou igual a 2) de servidores de controle de acesso. A Figura~\ref{fig:pervasive_hospital1} mostra uma possível distribuição de pontos de acesso e servidores de controle de acesso em um ambiente médico ubíquo. Este esquema pode incluir esquemas de tolerância a falhas de segurança que distribuem os certificados digitais dos pontos de acesso entre mais de um servidor de controle de acesso~\cite{rabin1989efficient}, de forma que se um servidor (ou mais, dependendo de $k$) forem atacados com sucesso, ainda não será possível obter o certificado necessário para acessar a base de dados dos pacientes.

\begin{figure}[htbp]
	\centering
		\includegraphics[width=.6\textwidth]{figuras/pervasive_hospital1.pdf}
	\caption{Possível distribuição de pontos de acesso e servidores de controle de acesso em um ambiente médico ubíquo. Círculos pequenos vazados representam servidores de controle de acesso enquanto os círculos maiores representam o alcance de seu sinal. Círculos cheios representam pontos de acesso que ``alcançam'' três servidores, e quadrados cheios representam pontos de acesso que ``alcançam'' menos de três servidores e podem não obter credenciais para acessar o banco.}
	\label{fig:pervasive_hospital1}
\end{figure}

Em um ambiente médico ubíquo, assim como em outros ambientes de computação, é extremamente importante proteger a rede e os computadores do ambiente de uma série de possíveis ataques, como ataques de negação de serviço (\emph{Denial of Service} -- DoS), hackerismos, cavalos de Troia, etc. Serviços de auditoria automática de segurança devem incluir mecanismos básicos~\cite{lunt1988automated,tsudik1990audes}, como detecção de intrusos ou um \emph{firewall}. Em particular, os mecanismos devem ser aprimorados para proteger uma infraestrutura altamente \textbf{heterogênea} e \textbf{distribuída}, como um sistema ubíquo. Por exemplo, um sistema distribuído de detecção de intrusos como proposto em~\cite{bass2000intrusion} consegue lidar com diversos tipos de ataques distribuídos. O próprio serviço de auditoria deve possuir suporte de tolerância a falhas, o que inclui suporte a operações desconectadas~\cite{kistler1992disconnected} para manter o serviço em desconexões transientes. Por exemplo, sensores distribuídos podem ter que \emph{bufferizar} dados durante o período de desconexão da rede.

Além destes fatores, decisões de arquitetura devem ser tomadas de forma aumentar, robutez, escalabilidade e autonomia do sistema médico ubíquo. A ideia é permitir que o sistema seja adaptativo de tal forma que diferentes regiões físicas do sistema funcionem de maneira independente. Por exemplo, um incêncio em uma ala de um hospital, resultando em perdas de pontos e servidores de acesso ubíquo, não deve fazer com que o sistema inteiro interrompa os serviços. Além disso, expandir o sistema deve ser uma tarefa simples.

% subsection questões_de_medicina (end)

Soluções desenvolvidos em outras sub-áreas da computação podem ser interessantes para a computação ubíqua, por exemplo, em~\cite{parikhformally} o autor apresenta um protótipo de arquitetura many-core com NoC (\emph{Network on Chip}) e roteadores autorreconfiguráveis e tolerantes a falhas em nível de \emph{hardware}.


\section{Conclusões}
\label{sec:conclusoes}
Tolerância a falhas é um assunto ainda pouco maduro na área computação ubíqua, em parte devido ao fato da área da computação ubíqua em si ser relativamente nova. Contudo, avanços significativos têm sido alcançados, alguns desses avanços diluídos em aplicações específicas de computação ubíqua que necessitam um ou outro recurso de tolerância a falhas. A maioria dos trabalhos estudados busca identificar um conjunto de técnicas básicas de tolerância a falhas que qualquer sistema de computação ubíqua devem implementar e empacotálas em forma de um \emph{framework} ou um \emph{middleware} para suporte ao desenvolvimento de aplicações ubíquas.

A área da computação ubíqua tem um perfil mais focado na interface do sistema distribuído com o usuário final e na incremental fusão de elementos de computação com ambiente físico. Contudo, para que a ubiquidade seja viável é primordial que o sistema seja tolerante a falhas, pois as pessoas dependerão destes sistemas para tarefas complexas e, muitas vezes, vitais. As ferramentas e técnicas de tolerância a falhas necessitam ainda de um considerável amadurecimento, que certamente se acentuará conforme houver mais demanda comercial por soluções de computação ubíqua.

% \begin{center}
% 	\textbf{Agradecimentos}
% \end{center}
% Gostaria de fazer um agradecimento especial aos irmãos Maurício e Laércio Lima Pilla, pelo contrabando de artigos.
	

\bibliographystyle{sbc}
\bibliography{bibs}

\end{document}